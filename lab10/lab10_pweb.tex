%\packagelist
\documentclass{article}
\usepackage[top=3cm, bottom=3cm, outer=3cm, inner=3cm]{geometry}
\usepackage{multicol}
\usepackage{graphicx}
\graphicspath{{img/}}
\usepackage{url}
%\usepackage{cite}
\usepackage{hyperref}
\usepackage{array}
%\usepackage{multicol}
\newcolumntype{x}[1]{>{\centering\arraybackslash\hspace{0pt}}p{#1}}
\usepackage{natbib}
\usepackage{pdfpages}
\usepackage{multirow}
\usepackage[normalem]{ulem}
\useunder{\uline}{\ul}{}
\usepackage{svg}
\usepackage{xcolor}
\usepackage{listings}
\lstset{basicstyle=\ttfamily,
  showstringspaces=false,
  commentstyle=\color{red},
  keywordstyle=\color{blue}
}
%\usepackage{booktabs}
\usepackage{caption}
\usepackage{subcaption}
\usepackage{float}
\usepackage{array}

\newcolumntype{M}[1]{>{\centering\arraybackslash}m{#1}}
\newcolumntype{N}{@{}m{0pt}@{}}


%%%%%%%%%%%%%%%%%%%%%%%%%%%%%%%%%%%%%%%%%%%%%%%%%%%%%%%%%%%%%%%%%%%%%%%%%%%%
%%%%%%%%%%%%%%%%%%%%%%%%%%%%%%%%%%%%%%%%%%%%%%%%%%%%%%%%%%%%%%%%%%%%%%%%%%%%
\newcommand{\itemEmail}{lsequeiros@unsa.edu.pe}
\newcommand{\itemStudent}{Luis Gustavo Sequeiros Condori}
\newcommand{\itemCourse}{Programación Web I}
\newcommand{\itemSemester}{II}
\newcommand{\itemUniversity}{Universidad Nacional de San Agustín de Arequipa}
\newcommand{\itemFaculty}{Facultad de Ingeniería de Producción y Servicios}
\newcommand{\itemDepartment}{Departamento Académico de Ingeniería de Sistemas e Informática}
\newcommand{\itemSchool}{Escuela Profesional de Ingeniería de Sistemas}
\newcommand{\itemAcademic}{2023 - B}
\newcommand{\itemInput}{Del 3 Enero 2024}
\newcommand{\itemOutput}{Al 6 Enero 2024}
\newcommand{\itemPracticeNumber}{10}
\newcommand{\itemTheme}{Bases de Datos MySQL}
%%%%%%%%%%%%%%%%%%%%%%%%%%%%%%%%%%%%%%%%%%%%%%%%%%%%%%%%%%%%%%%%%%%%%%%%%%%%
%%%%%%%%%%%%%%%%%%%%%%%%%%%%%%%%%%%%%%%%%%%%%%%%%%%%%%%%%%%%%%%%%%%%%%%%%%%%

\usepackage[english,spanish]{babel}
\usepackage[utf8]{inputenc}
\AtBeginDocument{\selectlanguage{spanish}}
\renewcommand{\figurename}{Figura}
\renewcommand{\refname}{Referencias}
\renewcommand{\tablename}{Tabla} %esto no funciona cuando se usa babel
\AtBeginDocument{%
	\renewcommand\tablename{Tabla}
}

\usepackage{fancyhdr}
\pagestyle{fancy}
\fancyhf{}
\setlength{\headheight}{30pt}
\renewcommand{\headrulewidth}{1pt}
\renewcommand{\footrulewidth}{1pt}
\fancyhead[L]{\raisebox{-0.2\height}{\includegraphics[width=3cm]{logo_episunsa.png}}}
\fancyhead[C]{\fontsize{7}{7}\selectfont	\itemUniversity \\ \itemFaculty \\ \itemDepartment \\ \itemSchool \\ \textbf{\itemCourse}}
\fancyhead[R]{\raisebox{-0.2\height}{\includegraphics[width=1.2cm]{logo_abet.png}}}
\fancyfoot[R]{Página \thepage}

% para el codigo fuente
\usepackage{listings}
\usepackage{color, colortbl}
\definecolor{numberOrange}{RGB}{235, 126, 9}
\definecolor{mygreen}{RGB}{40, 194, 6}
\definecolor{dkgreen}{rgb}{0,0.6,0}
\definecolor{myblue}{RGB}{57, 118, 189}
\definecolor{gray}{rgb}{0.156,0.146,0.135}
\definecolor{mygray}{RGB}{135, 135, 135}
\definecolor{mauve}{rgb}{0.58,0,0.82}
\definecolor{codebackgroundCode}{RGB}{10, 10, 20}
\definecolor{codebackgroundBash}{rgb}{0.95, 0.95, 0.92}
\definecolor{tablebackground}{RGB}{10, 25, 115}
\definecolor{lines}{RGB}{0, 120, 250}

\arrayrulecolor{lines}
\lstdefinestyle{custom}{
  %Las líneas que encierran al código: top, below, left, right. COn mayúsculas son dos líneas
  frame=tlrb,
  %Los espacios en los strings s muestran con normalidad, no con carácter especial
  showstringspaces=false,
  %no tiene cuidado del alineamiento de columnas, contrario a fixed, con ese valor es más riguroso
  columns=flexible,
  %Permite saltos de línea cuando el texto es muy largo
  breaklines=true,
  %Luego de un salto de línea automático, se coloca una flecita roja
  postbreak=\mbox{\textcolor{red}{$\hookrightarrow$}\space},
  %Los saltos de línea solo deben ocurrir en espacios en blanco
  breakatwhitespace=true,
  %espacios de tabulador
  tabsize=2,
  %No mostrar tabulador como caracter, como un espacio y ya
  showtabs=false,
  %No mostrar los espacios como caracteres
  showspaces=false,
  %No mostrar líneas al final de los listados
  showlines=false,
  %La codificación del archivo
  inputencoding=utf8,
  %Permite manejar los caracteres especiales junto con inputencoding  el paquete inputenc
  extendedchars=true,
  %
  literate={á}{\'a}1 {é}{\'e}1 {í}{\'i}1 {ó}{\'o}1 {ú}{\'u}1 {¿}{\textquestiondown}1 {ñ}{\~n}1 {Ñ}{\~N}1 
  {Á}{\'A}1 {É}{\'E}1 {Í}{\'I}1 {Ó}{\'O}1 {Ú}{\'U}1 {¡}{\textexclamdown}1
}

\lstdefinestyle{perl}{
  style=custom,
	language=Perl,
	basicstyle={\footnotesize\ttfamily\color[RGB]{255,255,255}},
	numberstyle=\color{mygray},
	numbers=left, 
  framexleftmargin=8mm,
  framexrightmargin=8mm,
  keywordstyle={\color{myblue}\itshape},
	%morekeywords={String, System, List, ArrayList, LinkedList, Scanner, Map, HashMap, TreeMap, Scanner},
  commentstyle={\color{mygray}\itshape},
  identifierstyle=\color[RGB]{16,151,228},
	stringstyle=\color{mygreen},
  %emph={int, char, boolean, String, double, float, byte, long, short, Integer, Character, Soldado, Boolean, Double, Float, Byte, Long, Short, List, Scanner, Map},
  emphstyle={\color[RGB]{244,151,32}},
	backgroundcolor= \color{codebackgroundCode}
}
\lstdefinestyle{html}{
  style=custom,
  language=HTML,
	basicstyle={\footnotesize\ttfamily\color[RGB]{255,255,255}},
	numberstyle=\color{mygray},
	numbers=left, 
  framexleftmargin=8mm,
  framexrightmargin=8mm,
  keywordstyle={\color{myblue}\itshape},
	%morekeywords={String, System, List, ArrayList, LinkedList, Scanner, Map, HashMap, TreeMap, Scanner},
  commentstyle={\color{mygray}\itshape},
  identifierstyle=\color[RGB]{16,151,228},
	stringstyle=\color{mygreen},
  %emph={int, char, boolean, String, double, float, byte, long, short, Integer, Character, Soldado, Boolean, Double, Float, Byte, Long, Short, List, Scanner, Map},
  %emphstyle={\color[RGB]{244,151,32}},
	backgroundcolor= \color{codebackgroundCode}
}
\lstdefinestyle{css}{
  style=custom,
  language=CSS,
	basicstyle={\footnotesize\ttfamily\color[RGB]{255,255,255}},
	numberstyle=\color{mygray},
	numbers=left, 
  framexleftmargin=8mm,
  framexrightmargin=8mm,
  keywordstyle={\color{myblue}\itshape},
	%morekeywords={String, System, List, ArrayList, LinkedList, Scanner, Map, HashMap, TreeMap, Scanner},
  commentstyle={\color{mygray}\itshape},
  identifierstyle=\color[RGB]{16,151,228},
	stringstyle=\color{mygreen},
  %emph={int, char, boolean, String, double, float, byte, long, short, Integer, Character, Soldado, Boolean, Double, Float, Byte, Long, Short, List, Scanner, Map},
  %emphstyle={\color[RGB]{244,151,32}},
	backgroundcolor= \color{codebackgroundCode}
}
\lstdefinestyle{mybash}{
  style=custom,
	language=bash,
	keepspaces=true,
	basicstyle={\small\ttfamily},
	keywordstyle=\color{myblue},
	morekeywords={java, javac, git},
	commentstyle=\color{mygray},
	stringstyle=\color{mygreen},
	backgroundcolor= \color{codebackgroundBash}
}

\begin{document}
	
	\vspace*{10px}
	
	\begin{center}	
		\fontsize{17}{17} \textbf{ Informe de Laboratorio \itemPracticeNumber}
	\end{center}
	\centerline{\textbf{\Large Tema: \itemTheme}}
	\vspace*{0.5cm}	

	\begin{table}[H]
		\begin{tabular}{|x{4.7cm}|x{4.8cm}|x{4.8cm}|}
			\hline 
			\rowcolor{tablebackground}
			\color{white} \textbf{Estudiante} & \color{white}\textbf{Escuela}  & \color{white}\textbf{Asignatura}   \\
			\hline 
      {\itemStudent \par \itemEmail} & \itemSchool & {\itemCourse \par Semestre: \itemSemester}     \\
			\hline 			
		\end{tabular}
	\end{table}		
	
	\begin{table}[H]
		\begin{tabular}{|x{4.7cm}|x{4.8cm}|x{4.8cm}|}
			\hline 
			\rowcolor{tablebackground}
			\color{white}\textbf{Semestre académico} & \color{white}\textbf{Fecha de inicio}  & \color{white}\textbf{Fecha de entrega}   \\
			\hline 
			\itemAcademic & \itemInput &  \itemOutput  \\
			\hline 
		\end{tabular}
	\end{table}

	\section{Objetivos}
	\begin{itemize}		
    \item Conocer las bases de datos
    \item Aprender a utilizar las bases de datos a través del SGBD MariaDB
    \item Valorar la potencialidad de los Sistemas Gestores de BD
	\end{itemize}
		
	\section{Temas a Tratar:}
	\begin{itemize}
		\item Expresiones regulares en Perl
    \item CGI en Perl
    \item HTML
    \item CSS
    \item Consultas a Bases de Datos con MariaDB
    \item Comandos básicos SQL
	\end{itemize}
	
	\section{Ejercicios Propuestos}

  \subsection{Archivo \textit{consult.html}}

  \lstinputlisting[style=html, caption={Estructura de la página de consulta}]{consult.html}

  \begin{quote}
    Una estructura simple con una envoltura para toda la página. Posee un título, 3 opciones que representan los primeros tres ejercicios y un formulario para el cuarto ejercicio, el cual se relaciona con \textit{fourth.pl}. Traté de hacer una estructura de tarjetas para que se vea ordenado y limpio. 
  \end{quote}

  \subsection{Archivo \textit{styles.css}}

  \lstinputlisting[style=html, caption={Hoja de estilos general}]{styles.css}

  \begin{quote}
    Se empieza una una normalización simple a los elementos y una un poco más especial para las referencias y los contenedores (utilizo flexbox). Tambień se define una fuente para la página. Posteriormente se definen algunos colores que se utilizará.
    Empezando con el diseño en general, se configura una envoltura para toda la página, que tenga sus elementos centrados. Luego, se añade un título para hacer la página más llamativa.
    Aquí, nos encontramos con el diseño de los contenedores,\textit{container} para un contenedor de contenidos, y \textit{content} para el contenido en sí mismo. Se trata de dar un formato de tarjetas con bordes redondeados y al centro, esto para el caso de los primeros tres ejercicios. Para el caso del último, se incluye el diseño del \textit{form}, de manera que la interfaz de ingreso de datos sea llamativa y senciilla para el usuario.
    Nos encontramos con el diseño para las respuestas del CGI. Dentro de un contenedor \textit{content}, se encuentra otro contenedor llamado \textit{mytable}, el cual fue creado para manejar el desbordamiento de la tabla, tanto horizontal como verticalmente.
    Por último tenemos el diseño de los botones volver.
  \end{quote}

  \subsection{Archivos CGI}

  \lstinputlisting[style=html, caption={CGI del primer ejercicio}]{cgi-bin/first.pl}

  \begin{quote}
  Al comienzo del archivo se puede observar la estructura de la página a mostrar, la cual es prácticamente la misma que \textit{consult.html}, así como la creación del objeto cgi. Luego, se conecta a la base de datos MariaDB con el y se guarda la referencia en \textit{dbh}. Aquí se hace uso de la biblioteca  \textit{DBI} y \textit{Net::Address::IP::Local}, con esta última se obtiene el IP del servidor.
  Luego se prepara y ejecuta una inserción en la base de datos, en este caso del ActorID 5. A continuación, se obtienen los campos (columnas) para incluirlo en la página a mostrar. 
  Nuevamente se prepara la referencia de la base de datos para obtener los datos del actor de ID 5, luego se ejecuta la petición, los registros son obtenidos uno a uno en \textit{row}, para mostrarlos en el HTML.
  Luego simplemente se muestra un botón para volver a la página web de consulta.
  \end{quote}

  \lstinputlisting[style=html, caption={CGI del segundo ejercicio}]{cgi-bin/second.pl}

  \begin{quote}
    Este CGI se ejecuta de la misma manera que el anterior, pero esta vez muestra las películas del año 1985, por lo tanto los campos que muestra son de la tabla \textit{Movie}.
  \end{quote}

  \lstinputlisting[style=html, caption={CGI del tercer ejercicio}]{cgi-bin/third.pl}

  \begin{quote}
    Se ejecuta como en los casos anteriores, pero esta vez muestra las películas con un Score mayor a 7 y Votos mayores a 5000. Lo único que cambia es la sentencia de la petición de datos.
  \end{quote}

  \lstinputlisting[style=html, caption={CGI del cuarto ejercicio}]{cgi-bin/fourth.pl}

  \begin{quote}
    Para este último CGI, la interacción con el servidor es la misma, pero, esta vez se obtienen los datos de un formulario y con el método \textit{get}. En este caso, se ingresa un año y el CGI realiza la petición a la base de datos con el año ingresado.
  \end{quote}

  \subsection{Recursos}

  \begin{itemize}
    \item Video sobre el funcionamiento de la página \url{https://youtu.be/6ISN8tj86vQ}.
    \item Repositorio de GitHub para commits \url{https://github.com/gusCreator/pweb-course/tree/main/lab10}.
  \end{itemize}

\clearpage
	
%\clearpage
%\bibliographystyle{apalike}
%\bibliographystyle{IEEEtranN}
%\bibliography{bibliography}
			
\end{document}
