%\packagelist
\documentclass{article}
\usepackage[top=3cm, bottom=3cm, outer=3cm, inner=3cm]{geometry}
\usepackage{multicol}
\usepackage{graphicx}
\graphicspath{{img/}}
\usepackage{url}
%\usepackage{cite}
\usepackage{hyperref}
\usepackage{array}
%\usepackage{multicol}
\newcolumntype{x}[1]{>{\centering\arraybackslash\hspace{0pt}}p{#1}}
\usepackage{natbib}
\usepackage{pdfpages}
\usepackage{multirow}
\usepackage[normalem]{ulem}
\useunder{\uline}{\ul}{}
\usepackage{svg}
\usepackage{xcolor}
\usepackage{listings}
\lstloadlanguages{HTML}
\lstloadlanguages{CSS}
\lstset{basicstyle=\ttfamily,
  showstringspaces=false,
  commentstyle=\color{red},
  keywordstyle=\color{blue}
}
%\usepackage{booktabs}
\usepackage{caption}
\usepackage{subcaption}
\usepackage{float}
\usepackage{array}

\newcolumntype{M}[1]{>{\centering\arraybackslash}m{#1}}
\newcolumntype{N}{@{}m{0pt}@{}}


%%%%%%%%%%%%%%%%%%%%%%%%%%%%%%%%%%%%%%%%%%%%%%%%%%%%%%%%%%%%%%%%%%%%%%%%%%%%
%%%%%%%%%%%%%%%%%%%%%%%%%%%%%%%%%%%%%%%%%%%%%%%%%%%%%%%%%%%%%%%%%%%%%%%%%%%%
\newcommand{\itemEmail}{lsequeiros@unsa.edu.pe}
\newcommand{\itemStudent}{Luis Gustavo Sequeiros Condori}
\newcommand{\itemCourse}{Programación Web I}
\newcommand{\itemSemester}{II}
\newcommand{\itemUniversity}{Universidad Nacional de San Agustín de Arequipa}
\newcommand{\itemFaculty}{Facultad de Ingeniería de Producción y Servicios}
\newcommand{\itemDepartment}{Departamento Académico de Ingeniería de Sistemas e Informática}
\newcommand{\itemSchool}{Escuela Profesional de Ingeniería de Sistemas}
\newcommand{\itemAcademic}{2023 - B}
\newcommand{\itemInput}{Del 23 Diciembre 2023}
\newcommand{\itemOutput}{Al 27 Diciembre 2023}
\newcommand{\itemPracticeNumber}{07}
\newcommand{\itemTheme}{Perl}
%%%%%%%%%%%%%%%%%%%%%%%%%%%%%%%%%%%%%%%%%%%%%%%%%%%%%%%%%%%%%%%%%%%%%%%%%%%%
%%%%%%%%%%%%%%%%%%%%%%%%%%%%%%%%%%%%%%%%%%%%%%%%%%%%%%%%%%%%%%%%%%%%%%%%%%%%

\usepackage[english,spanish]{babel}
\usepackage[utf8]{inputenc}
\AtBeginDocument{\selectlanguage{spanish}}
\renewcommand{\figurename}{Figura}
\renewcommand{\refname}{Referencias}
\renewcommand{\tablename}{Tabla} %esto no funciona cuando se usa babel
\AtBeginDocument{%
	\renewcommand\tablename{Tabla}
}

\usepackage{fancyhdr}
\pagestyle{fancy}
\fancyhf{}
\setlength{\headheight}{30pt}
\renewcommand{\headrulewidth}{1pt}
\renewcommand{\footrulewidth}{1pt}
\fancyhead[L]{\raisebox{-0.2\height}{\includegraphics[width=3cm]{logo_episunsa.png}}}
\fancyhead[C]{\fontsize{7}{7}\selectfont	\itemUniversity \\ \itemFaculty \\ \itemDepartment \\ \itemSchool \\ \textbf{\itemCourse}}
\fancyhead[R]{\raisebox{-0.2\height}{\includegraphics[width=1.2cm]{logo_abet.png}}}
\fancyfoot[R]{Página \thepage}

% para el codigo fuente
\usepackage{listings}
\usepackage{color, colortbl}
\definecolor{numberOrange}{RGB}{235, 126, 9}
\definecolor{mygreen}{RGB}{40, 194, 6}
\definecolor{dkgreen}{rgb}{0,0.6,0}
\definecolor{myblue}{RGB}{57, 118, 189}
\definecolor{gray}{rgb}{0.156,0.146,0.135}
\definecolor{mygray}{RGB}{135, 135, 135}
\definecolor{mauve}{rgb}{0.58,0,0.82}
\definecolor{codebackgroundCode}{RGB}{10, 10, 20}
\definecolor{codebackgroundBash}{rgb}{0.95, 0.95, 0.92}
\definecolor{tablebackground}{RGB}{10, 25, 115}
\definecolor{lines}{RGB}{0, 120, 250}

\arrayrulecolor{lines}
\lstdefinestyle{custom}{
  %Las líneas que encierran al código: top, below, left, right. COn mayúsculas son dos líneas
  frame=tlrb,
  %Los espacios en los strings s muestran con normalidad, no con carácter especial
  showstringspaces=false,
  %no tiene cuidado del alineamiento de columnas, contrario a fixed, con ese valor es más riguroso
  columns=flexible,
  %Permite saltos de línea cuando el texto es muy largo
  breaklines=true,
  %Luego de un salto de línea automático, se coloca una flecita roja
  postbreak=\mbox{\textcolor{red}{$\hookrightarrow$}\space},
  %Los saltos de línea solo deben ocurrir en espacios en blanco
  breakatwhitespace=true,
  %espacios de tabulador
  tabsize=2,
  %No mostrar tabulador como caracter, como un espacio y ya
  showtabs=false,
  %No mostrar los espacios como caracteres
  showspaces=false,
  %No mostrar líneas al final de los listados
  showlines=false,
  %La codificación del archivo
  inputencoding=utf8,
  %Permite manejar los caracteres especiales junto con inputencoding  el paquete inputenc
  extendedchars=true,
  %
  literate={á}{\'a}1 {é}{\'e}1 {í}{\'i}1 {ó}{\'o}1 {ú}{\'u}1 {¿}{\textquestiondown}1 {ñ}{\~n}1 {Ñ}{\~N}1 
  {Á}{\'A}1 {É}{\'E}1 {Í}{\'I}1 {Ó}{\'O}1 {Ú}{\'U}1 {¡}{\textexclamdown}1
}

\lstdefinestyle{perl}{
  style=custom,
	language=Perl,
	basicstyle={\footnotesize\ttfamily\color[RGB]{255,255,255}},
	numberstyle=\color{mygray},
	numbers=left, 
  framexleftmargin=8mm,
  framexrightmargin=8mm,
  keywordstyle={\color{myblue}\itshape},
	%morekeywords={String, System, List, ArrayList, LinkedList, Scanner, Map, HashMap, TreeMap, Scanner},
  commentstyle={\color{mygray}\itshape},
  identifierstyle=\color[RGB]{16,151,228},
	stringstyle=\color{mygreen},
  %emph={int, char, boolean, String, double, float, byte, long, short, Integer, Character, Soldado, Boolean, Double, Float, Byte, Long, Short, List, Scanner, Map},
  emphstyle={\color[RGB]{244,151,32}},
	backgroundcolor= \color{codebackgroundCode}
}
\lstdefinestyle{html}{
  style=custom,
  language=HTML,
	basicstyle={\footnotesize\ttfamily\color[RGB]{255,255,255}},
	numberstyle=\color{mygray},
	numbers=left, 
  framexleftmargin=8mm,
  framexrightmargin=8mm,
  keywordstyle={\color{myblue}\itshape},
	%morekeywords={String, System, List, ArrayList, LinkedList, Scanner, Map, HashMap, TreeMap, Scanner},
  commentstyle={\color{mygray}\itshape},
  identifierstyle=\color[RGB]{16,151,228},
	stringstyle=\color{mygreen},
  %emph={int, char, boolean, String, double, float, byte, long, short, Integer, Character, Soldado, Boolean, Double, Float, Byte, Long, Short, List, Scanner, Map},
  %emphstyle={\color[RGB]{244,151,32}},
	backgroundcolor= \color{codebackgroundCode}
}
\lstdefinestyle{css}{
  style=custom,
  language=CSS,
	basicstyle={\footnotesize\ttfamily\color[RGB]{255,255,255}},
	numberstyle=\color{mygray},
	numbers=left, 
  framexleftmargin=8mm,
  framexrightmargin=8mm,
  keywordstyle={\color{myblue}\itshape},
	%morekeywords={String, System, List, ArrayList, LinkedList, Scanner, Map, HashMap, TreeMap, Scanner},
  commentstyle={\color{mygray}\itshape},
  identifierstyle=\color[RGB]{16,151,228},
	stringstyle=\color{mygreen},
  %emph={int, char, boolean, String, double, float, byte, long, short, Integer, Character, Soldado, Boolean, Double, Float, Byte, Long, Short, List, Scanner, Map},
  %emphstyle={\color[RGB]{244,151,32}},
	backgroundcolor= \color{codebackgroundCode}
}
\lstdefinestyle{mybash}{
  style=custom,
	language=bash,
	keepspaces=true,
	basicstyle={\small\ttfamily},
	keywordstyle=\color{myblue},
	morekeywords={java, javac, git},
	commentstyle=\color{mygray},
	stringstyle=\color{mygreen},
	backgroundcolor= \color{codebackgroundBash}
}

\begin{document}
	
	\vspace*{10px}
	
	\begin{center}	
		\fontsize{17}{17} \textbf{ Informe de Laboratorio \itemPracticeNumber}
	\end{center}
	\centerline{\textbf{\Large Tema: \itemTheme}}
	\vspace*{0.5cm}	

	\begin{table}[H]
		\begin{tabular}{|x{4.7cm}|x{4.8cm}|x{4.8cm}|}
			\hline 
			\rowcolor{tablebackground}
			\color{white} \textbf{Estudiante} & \color{white}\textbf{Escuela}  & \color{white}\textbf{Asignatura}   \\
			\hline 
      {\itemStudent \par \itemEmail} & \itemSchool & {\itemCourse \par Semestre: \itemSemester}     \\
			\hline 			
		\end{tabular}
	\end{table}		
	
	\begin{table}[H]
		\begin{tabular}{|x{4.7cm}|x{4.8cm}|x{4.8cm}|}
			\hline 
			\rowcolor{tablebackground}
			\color{white}\textbf{Semestre académico} & \color{white}\textbf{Fecha de inicio}  & \color{white}\textbf{Fecha de entrega}   \\
			\hline 
			\itemAcademic & \itemInput &  \itemOutput  \\
			\hline 
		\end{tabular}
	\end{table}

	%\section{Objetivos}
	%\begin{itemize}		
   % \item 
	%\end{itemize}
		
	\section{Temas a Tratar:}
	\begin{itemize}
		\item CGI en perl
	\end{itemize}
	
	\section{Actividades}

  \subsection{Archivo \textit{simple-se.html}}

  \lstinputlisting[style=html, caption={Esta es la estructura de la página de búsqueda simple}]{simple-se.html}

  \begin{quote}
    Se puede observar la estructura en el \textit{head} por defecto que debería tener todo documento HTML. Luego, se observa la estructura del la barra de navegación, con tres botones que se dirigen a las respectivas páginas creadas. Posteriormente, se observa la imagen del logo de la página, al estilo de Google; asimismo, el \textit{form} que permite la entrada de datos para búsqueda. También, se encuentra el botón para iniciar la ejecución del CGI respectivo que redireccionará la página hacia la búsqueda simple en Google.
  \end{quote}

  \subsection{Archivo \textit{advanced-se.html}}

  \lstinputlisting[style=html, caption={Estructura para la página de búsqueda avanzada}]{advanced-se.html}

  \begin{quote}
    Se observa la misma estructura para el head y la barra de navegación, así como el logo. Lo que cambia aquí son los campos para ingresar datos, cada uno tiene indicaciones para realizar una búsqueda más avanzada. Traté de hacer la estructura acorde a la búsqueda avanzada de Google, aunque me pareción poco llamativa.
  \end{quote}

  \subsection{Archivo \textit{image-se.html}}

  \lstinputlisting[style=html, caption={Estructura para la página de búsqueda de imágenes}]{image-se.html}

  \begin{quote}
    Esta estructura es la misma que la búsqueda simple, solo cambia el CGi que se utiliza, para redireccionar la búsqueda en imágenes con Google.
  \end{quote}

  \subsection{Archivo \textit{styles.css}}

  \lstinputlisting[style=css, caption={Hoja de estilos para todas las páginas}]{styles.css}

  \begin{quote}
    La hoja de estilos utiiza el color azul característico de Google, así como el gris con el que combina. Se utilizan varios contenedores flex para hacer más fácil el centrar elementos y acomodarlos. Se trata de hacer diseño responsivo aunque no es para nada perfecto. En esta hoja de estilos es donde se realizan los bordes redondeados de los campos de entrada y de los botones.
  \end{quote}

  \subsection{Los CGI}

  \lstinputlisting[style=perl, caption={CGI para la búsqueda simple}]{cgi-bin/simple-se.pl}
  \lstinputlisting[style=perl, caption={CGI para la búsqueda avanzada}]{cgi-bin/advanced-se.pl}
  \lstinputlisting[style=perl, caption={CGI para la búsqueda de imágenes}]{cgi-bin/image-se.pl}

  \begin{quote}
    Los tres CGI son lo mismo en si, lo único que se diferencian es el redireccionamiento. cada uno de acuerdo a lo obtenido en los forms de HTML. Redirecciona la página hacia la de búsqueda.
  \end{quote}

  \begin{quotation}
    Para revisar más detalles del desarrollo de este laboratorio, puede revisar el siguiente repositorio: \url{https://github.com/gusCreator/pweb-course/tree/main/lab07}. Además, adjunto el enlace para YouTube: \url{https://www.youtube.com/watch?v=-EzRPkyupmI}
  \end{quotation}

\clearpage
	
%\clearpage
%\bibliographystyle{apalike}
%\bibliographystyle{IEEEtranN}
%\bibliography{bibliography}
			
\end{document}
